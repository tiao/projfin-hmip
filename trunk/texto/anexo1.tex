% ----------------------------------------------------------------------- %
% Arquivo: anexo1.tex
% ----------------------------------------------------------------------- %

\chapter{Gerador de tr�fego}
\label{c_anexo1}

\begin{verbatim}
 /*
* gen-0.1
*
* (c) 2008 Cleiber Marques
*
* This program is free software; you can redistribute it and /or
* modify it under the terms of the GNU General Public License
* as published by the Free Software Foundation; either version
* 2 of the License, or (at your option) any later version.
*
*/

#include <stdio.h>
#include <stdlib.h>
#include <unistd.h>
#include <errno.h>
#include <string.h>
#include <time.h>
#include <sys/types.h>
#include <sys/socket.h>
#include <netinet/in.h>
#include <arpa/inet.h>
#include <netdb.h>

#define PORT 4950
#define SIZE_PACK 100
#define IPV6 
/*#define LOOP*/

int main(int argc, char **argv)
{
  int fd, numbytes, len;
  char buf[SIZE_PACK];
#ifdef IPV6
  struct sockaddr_in6 gen, rcv;
#else
  struct sockaddr_in gen, rcv;
#endif
  struct tm *local;
  time_t t;
  char addr[INET6_ADDRSTRLEN];
  /* Cria um socket UDP */
#ifdef IPV6
  printf("Program version IPv6\n");
  if (!(fd = socket(AF_INET6, SOCK_DGRAM, 0))) {
#else
  printf("Program version IPv4\n");
  if (!(fd = socket(AF_INET, SOCK_DGRAM, 0))) {
#endif
    fprintf(stderr, "Error in create socket\n");
    exit (1);
  }
  bzero(&gen, sizeof(gen));
#ifdef IPV6
  gen.sin6_family = AF_INET6;
  gen.sin6_port = htons(PORT);
#else
  gen.sin_family = AF_INET;
  gen.sin_port = htons(PORT);
#endif
  if(argc == 1) {
#ifdef IPV6
    gen.sin6_addr = in6addr_any;
#else
    gen.sin_addr.s_addr = INADDR_ANY;
#endif
    /* Bind */
    if (bind(fd, (struct sockaddr *)&gen, sizeof(gen)) == -1) {
      fprintf(stderr, "Error in bind\n");
      exit(1);
    }
    while(1) {
      len = sizeof(rcv);
      /* Recebe pacote */
      if ((numbytes=recvfrom(fd, buf, SIZE_PACK , 0,
          (struct sockaddr *)&rcv, &len)) == -1) {
        fprintf(stderr,"Error in recvfrom\n");
        exit(1);
      }
      /* Pega a hora do sistema e imprime */
      t = time(NULL);
      local = localtime(&t);
#ifdef LOOP
      /* Envia pacote */
      if ((numbytes = sendto(fd, buf, SIZE_PACK, 0,
          (struct sockaddr *)&rcv, sizeof(gen))) == -1) 
        fprintf(stderr, "Error in sendto\n");
#endif
      /* Imprime log */
#ifdef IPV6
      inet_ntop(AF_INET6, &rcv.sin6_addr, addr, sizeof(addr));
      printf("Receive packet [%d:%d:%d] %d bytes from %s seq=%d\n", local->tm_hour,
        local->tm_min, local->tm_sec, numbytes, addr, buf[0]);
#else
      printf("Receive packet [%d:%d:%d] %d bytes from %s seq=%d\n", local->tm_hour,
        local->tm_min, local->tm_sec, numbytes, inet_ntoa(rcv.sin_addr), buf[0]);
#endif
    }
  } else {
    int count = 0, i;
#ifdef IPV6
    inet_pton(AF_INET6, argv[1], &gen.sin6_addr);
#else
    inet_pton(AF_INET, argv[1], &gen.sin_addr);
#endif
    while(1) {
      srand(time(NULL));

      /* Completa o pacote com numeros randomicos */
      for(i=0; i<SIZE_PACK; i++)
        buf[i] = rand() % 10; 
      /* Envia pacote */
      if ((numbytes = sendto(fd, buf, SIZE_PACK, 0,
          (struct sockaddr *)&gen, sizeof(gen))) == -1) {
        fprintf(stderr, "Error in sendto\n");
        exit(1);
      }
      /* Pega a hora do sistema e imprime  */
      t = time(NULL);
      local = localtime(&t);
#ifdef LOOP
      len = sizeof(rcv);
      /* Recebe pacote */
      if ((numbytes=recvfrom(fd, buf, SIZE_PACK , 0,
          (struct sockaddr *)&rcv, &len)) == -1) {
        fprintf(stderr,"Error in recvfrom\n");
        exit(1);
      }
#endif
      /* Imprime log */
      printf("Send packet [%d:%d:%d] %d bytes from %s seq=%d\n",
        local->tm_hour, local->tm_min, local->tm_sec, numbytes, argv[1], buf[0]);
      sleep(1);
    }
  }
}
\end{verbatim}
