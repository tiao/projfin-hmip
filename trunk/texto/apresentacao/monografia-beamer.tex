% Emerson Ribeiro de Mello (emerson_ml@yahoo.com.br)
\documentclass{beamer}
%% comente a linha acima e descomente a debaixo para gerar
%% l�minas para serem impressas
% \documentclass[handout]{beamer}

% Outras classes: [notes], [notes=only], [trans], [handout],
% [red], [compress], [draft], [class=article]

\mode<presentation>{
  %% Temas
  \usepackage{beamerthemeshadow}
%   \usepackage{beamerthemebars}
%   \usepackage[headheight=12pt,footheight=12pt]{beamerthemeboxes}
%   \usepackage{beamerthemeclassic}
%   \usepackage{beamerthemelined}
%   \usepackage{beamerthemeplain}
%   \usepackage[width=12pt,dark,tab]{beamerthemesidebar}
%   \usepackage{beamerthemesplit}
%   \usepackage{beamerthemetree}
%   \usepackage[bar]{beamerthemetree}
  
  \usepackage{pgf}
  \beamertemplatetransparentcovereddynamic
  \beamertemplateballitem
%   \beamertemplatefootpagenumber
}

\mode<handout>{
  % tema simples para ser impresso
  \usepackage[bar]{beamerthemetree}
  % Colocando um fundo cinza quando for gerar transpar�ncias para serem impressas
  % mais de uma transpar�ncia por p�gina
  \beamertemplatesolidbackgroundcolor{black!5}
}

\usepackage{amsmath,amssymb}
\usepackage[brazil]{varioref}
\usepackage[english,brazil]{babel}
\usepackage[latin1]{inputenc}
\usepackage{graphicx}
\usepackage{listings}
\usepackage{url}
\usepackage{colortbl}
% um outro tipo de fonte
% \usepackage{pslatex}

\beamertemplatetransparentcovereddynamic

\title[Plataforma para o Estudo de Mobilidade na Camada de Rede]{Plataforma para o Estudo de Mobilidade na Camada de Rede}
\author[Silva, Almeida]{%
  Cleiber Marques da Silva \and Filipe Medeiros de Almeida}

\institute[Instituto Federal, Ci�ncia e Tecnologia de Santa Catarina]{
	Instituto Federal, Ci�ncia e Tecnologia de Santa Catarina -- Campus S�o Jos�\\
	Curso Superior de Tecnologia em Sistemas de Telecomunica��es
}

% Se comentar a linha abaixo, ir� aparecer a data quando foi compilada a apresenta��o  
\date{Defesa Projeto Conclus�o de Curso, 2009}

\pgfdeclareimage[height=1cm]{logo}{../figs/logo}

% pode-se colocar o LOGO assim
\logo{\pgfuseimage{logo}}

% ou...
%\logo{\vbox{\hbox to 1cm{\hfil\pgfuseimage{ufsc}}\vskip0.1cm\hbox{\pgfuseimage{das}}}}

\begin{document}

\frame{\titlepage}

\section[Sum�rio]{}
\frame{\tableofcontents}

\AtBeginSection[]{
  \frame<handout:0>{
    \frametitle{Sum�rio}
    \tableofcontents[current,currentsection]
  }
}
\section{Introdu��o}

\subsection{Justificativas e Objetivos}

\frame{
	\frametitle{Justificativas}
	\begin{itemize}
		\item As tecnologias de comunica��o sem fio est�o em constante
evolu��o
		\item A Internet continua em r�pida expans�o
		\item Cada dia � maior n�mero de dispositivos m�veis capaz de
acessar a Internet
		\item H� uma necessidade na continuidade dos servi�os
	\end{itemize}
}

\subsection{}

 \frame{
	\frametitle{Objetivos}
}
\section{Protocolos de Mobilidade da Camada Rede}

\subsection{O Problema da Mobilidade na Camada de Rede}
\frame{
\frametitle{Problemas atuais...}
	\begin{itemize}[<+-| alert@+>]
		\item Funcionamento das redes IP;
		\item O endere�os IP definine a localiza��o geogr�fica de um hospedeiro;
		\item A Internet assumem que o um n� n�o muda seu endere�o IP durante uma
comunica��o;
		\item Quando o n� muda seu ponto de conex�o, n�o � possivel atualizar a rota
 para o novo destino (escalabilidade);
	\end{itemize}
}
\frame{
	\frametitle{Propostas da IETF}
	\begin{itemize}[<+-| alert@+>]
		\item \textit{Mobile Internet Protocol}(MIP) ou IP M�vel;
		\item Permitir os n�s movimentar-se, para diferentes subredes, continuando 
as comunica��es j� estabelecidas;
		\item MIPv4 ou MIPv6?
		\item Uma extens�o do MIPv6: \textit{Hierarchical Mobile Internet Protocol}
(HMIP) ou IP M�vel Hier�rquico;
	\end{itemize}
}

\subsection{Vis�o Geral dos protocolo de mobilidade}
\frame{
	\frametitle{Funcionamento MIPv6}
	\begin{itemize}[<+-| alert@+>]
		\item O \textit{Mobile Node}(MN) ou n� m�vel sempre est� acess�vel pelo 
mesmo endere�o IP, o \textit{Home Address}(HoA) ou endere�o domiciliar;
		\item N�o depende da tecnologia de acesso;
		\item Para cada rede visitada ser� configurado um \textit{Care-of Address}
(CoA);
		\item Ap�s detectar o movimento e configurar seu CoA, o n� m�vel realiza o
registro com o seu Agente Domiciliar;
\end{itemize}

\begin{columns}
\column{10cm}
		\begin{block}{\textit{Agente Domiciliar}}
		\begin{itemize}
			\item<5-| alert@5> O \textit{Home Agent}(HA) ou agente domiciliar �
responsavel em encaminhar os pacotes para o n� m�vel, quando est� na rede
visitada;
		\end{itemize}
		\end{block}
\end{columns}
	
}

\frame{
	\frametitle{O Procedimento de Handover}
	\begin{itemize}[<+-| alert@+>]
		\item \textit{Handover} Camada 2(enlace) e {Handover} Camada 3(rede);
		\item O \textit{handover} na camada rede, na arquitetura IPv6, envolve:
		\begin{enumerate}
			\item Descoberta de um novo roteador;
			\item Auto configura��o do CoA;
			\item Teste de duplicidade do CoA (DAD), e;
			\item Registro com o agente domiciliar e o n� correspondente;
		\end{enumerate}
	\end{itemize}
}

\pgfdeclareimage[height=6cm]{mipv6}{../figs/mipv6}
\frame{
	\frametitle{Modos de Funcionamento}
		\pgfuseimage{mipv6}
}

\pgfdeclareimage[height=6cm]{fluxo_mipv6}{../figs/fluxo_mipv6}
\frame{
	\frametitle{O Procedimento de Handover}
	\pgfuseimage{fluxo_mipv6}
}

\frame{
	\frametitle{Considera��o sobre Seguran�a}
	\begin{itemize}[<+-| alert@+>]
		\item O uso de BUs n�o autenticados � um s�rio problema de seguran�a;
		\item Prote��o no registro(BU) com o Agente Domiciliar, utilizando o
protocolo IPSec;
		\item � necess�rio uma chave secreta deve ser configurada nos dois n�s,
inviabilizando a seguran�a entre n� movel e n� correspondente;
		\item\textit{Return Routability Procedure}(RRP);
	\end{itemize}
}

\pgfdeclareimage[height=6cm]{rrp}{../figs/rrp}

\frame{
	\frametitle{O Procedimento de Return Routability}
	\pgfuseimage{rrp}
}

\frame{
	\frametitle{Problemas do MIPv6}
	\begin{itemize}[<+- | alert@+>]
		\item O MIPv6 apresenta alguns problemas que comprometem a sua
escalabilidade na \textit{Internet};
		\begin{itemize}
			\item<2- | alert@2> tempo na detec��o do movimento;
			\item<2- | alert@2> tempo na configura��o do novo endere�o na rede
visitada, o CoA;
			\item<2- | alert@2> tempo do registro com o seu Agente Domiciliar;
		\end{itemize}
		\item<3- | alert@3>Uma solu��o � minimizar os longos retardos no envio dos
BU, tratando a mobilidade global e local de formas distintas;
	\end{itemize}
}

\frame{
	\frametitle{Problemas do MIPv6}
\begin{columns}
\column{10cm}

\begin{block}{Mobilidade Local} 
	\begin{itemize}\item<1- | alert@1> � a que acontece dentro de um mesmo
 dom�nio.\end{itemize}
\end{block}

\begin{block}{Mobilidade Global} 
	\begin{itemize}\item<2- | alert@2> Ocorre entre dom�nios
diferentes\end{itemize}
\end{block}

\begin{block}{Dom�nio}
 \begin{itemize}\item<3- | alert@3> Dom�nio � uma dimens�o arbitr�ria, por
  exemplo, uma rede de uma empresa ou universidade com uma ou mais
 subredes.\end{itemize}
\end{block}

\end{columns}
}

\frame{
	\frametitle{Funcionamento IP M�vel Hierarquico}
\begin{itemize}[<+- | alert@+>]
	\item Introdu��o de um novo agente, o \textit{Mobile Ancor Point}(MAP)
	\item Quando o n� m�vel entra em um dom�nio com a presen�a de um MAP, ele
ir� gerar os endere�os RCoA e LCoA;
\end{itemize}

\begin{columns}
\column{10cm}
\begin{block}{RCoA}
 \begin{itemize}[<+- | alert@+>]
	\item que ser� baseado no prefixo da rede do MAP.
 \end{itemize}
\end{block}
 
\begin{block}{LCoA}
 \begin{itemize}[<+- | alert@+>]
\item	utiliza o mesmo prefixo anunciado pelo roteador padr�o.
 \end{itemize}
\end{block}
\end{columns}
}

\frame{
	\frametitle{Funcionamento IP M�vel Hierarquico}
\begin{itemize}[<+- | alert@+>]
  \item O funcionamento do Agente Domiciliar e dos n�s correspondentes permanece
        id�ntico ao do MIPv6.
  \item O MAP intercepta os pacotes destinados ao RCoA.
  \item O n� m�vel ir� atualizar suas associa��es:
  \begin{enumerate}[<+- | alert@+>]
    \item MAP = (RCoA com o LCoA);
    \item Agente Domiciliar = (RCoA com endere�o domiciliar)
    \item Correspondente(opcional) = (RCoA com endere�o domiciliar)
    \item Correspondente locais(opcional) = (LCoA com endere�o domiciliar)
  \end{enumerate}
\end{itemize}
}

\pgfdeclareimage[height=6cm]{hmip}{../figs/hmipv6}
\frame{
	\frametitle{Funcionamento IP M�vel Hierarquico}
	\pgfuseimage{hmip}
}
%\pgfdeclareimage[height=6cm]{intra_site}{figs/intra_site}
%\pgfdeclareimage[height=6cm]{inter_site}{figs/inter_site}


\section{Plataforma de Testes para Mobilidade em Redes IP}
\pgfdeclareimage[height=4cm]{uml}{../figs/uml}
\pgfdeclareimage[height=2.3cm]{line_uml}{../figs/line_uml}
\pgfdeclareimage[height=6cm]{vnuml}{../figs/vnuml}
\pgfdeclareimage[height=6cm]{guml4mip}{../figs/guml4mip}
\pgfdeclareimage[height=6cm]{guml4mip_func}{../figs/guml4mip_func}
\pgfdeclareimage[height=1.5cm]{guml4mip_logo}{../figs/guml4mip_logo}
\pgfdeclareimage[height=3cm]{uml_switch}{../figs/uml_switch}

\frame{
	\frametitle{Objetivos da Plataforma de Testes}
	O objetivo geral da plataforma de testes � prover ao usu�rio:
	\begin{itemize}[<+-| alert@+>]
	 \item Ambiente virtual para o estudo dos protocolos de
mobilidade da camada rede
	 \item Facilidade e rapidez na constru��o de cen�rios de
rede 
	 \item Facilidade de instala��o e uso da plataforma
	\end{itemize}
}

\frame{
	\frametitle{Descri��o Geral da Plataforma}
	Componentes presentes na Plataforma de Testes:
	\begin{itemize}[<+-| alert@+>]
	 \item M�quinas e \textit{switches} virtuais UML
	 \item Interface gr�fica para:
	 \begin{itemize}
 	  \item controlar os terminais UML
	  \item constru��o r�pida de cen�rios de rede
	  \item comandar a mobilidade
	 \end{itemize}
	 \item Protocolos da camada de rede:
	 \begin{itemize}
	  \item protocolos de mobilidade: MIPv6 e HMIPv6
	  \item protocolos de roteamento din�micos: RIP, OSPF
	  \item RADVD gerador de mensagens RA
	 \end{itemize}
	 \item Ferramentas para o estudo dos protocolos
	\end{itemize}
}

\subsection{Linux em Modo Usu�rio}

\frame{
	\frametitle{M�quinas Virtuais UML}
	\begin{itemize}[<+-| alert@+>]
	 \item UML � uma implementa��o que permite executar um
kernel Linux em um sistema hospedeiro como um processo de usu�rio
	 \item � uma arquitetura suportada pelo kernel Linux, assim
como i386, Sparc, ARM. A arquitetura � o pr�prio kernel Linux
	\end{itemize} 
	\pgfuseimage{uml}
}

\frame{
	\frametitle{Usando a UML}
	\begin{itemize}[<+-| alert@+>]
	 \item Para fazer o uso das m�quinas UML � necess�rio
somente um sistema de arquivos e um kernel compilado para esta aquitetura
	 \item Para a plataforma foi criado um sistema de arquivos
a partir o GNU/Linux Debian, os principais recursos presentes s�o:
	 \begin{itemize}
	  \item Implementa��es do MIPv6 e HMIPv6
	  \item Protocolos de roteamento din�mico com o pacote \textit{Zebra}
	  \item Utilit�rios de rede: tcpdump, ping, ip, tc, netcat, iptables
	  \item Suporte no kernel: mobilidade, Netfilter, QoS, IPSec
	 \end{itemize}
	 \item  Outros recursos podem ser facilmente instalados na
Plataforma
	\end{itemize} 
}

\frame{
	\frametitle{Usando a UML}
	\pgfuseimage{line_uml}
	\begin{columns}
    	 \column{4.5cm}
	 \begin{block}{Compilar kernel UML}
	  \$ export ARCH=um\\
	  \$ make defconfig\\
	  \$ make menuconfig\\
	  \$ make\\
	 \end{block}
	 \column{3.5cm}
	\end{columns}
}

\frame{
	\frametitle{UML x VMWare}
	\begin{itemize}[<+-| alert@+>]
	 \item UML � um sistema operacional virtual
	 \item Tecnologias de virtualiza��o como VMWare, simulam
uma plataforma f�sica, onde o sistema � executado
	 \item UML tem um alto desempenho em execu��o
	 \item UML tem problemas para utilizar recursos de
hardware, por exemplo PCI
	 \item UML permite iniciar maior n�mero de m�quinas
	\end{itemize}
}

\frame{
	\frametitle{Utiliza��o das M�quinas UML}
	Devido a sua flexibilidade as m�quinas UML podem ser utilizadas para
in�meras aplica��es, tais como:
	\begin{itemize}[<+-| alert@+>]
	 \item Consolida��o de servidores de rede
	 \item Ambientes de ensino
	 \item Desenvolvimento de software:
	 \begin{itemize}
	  \item aplica��es em n�vel usu�rio
	  \item protocolos de rede
	  \item pr�prio kernel linux
	 \end{itemize}
	\end{itemize}
}

\subsection{Redes Virtuais com UML}

\frame{
	\frametitle{Redes Virtuais com UML}
	Redes para as m�quinas UML:
	\begin{itemize}[<+-| alert@+>]
	 \item Conectando UML com o hospedeiro, utilizando as
tecnologias de transporte:
	 \begin{itemize}
	  \item TUN/TAP
	  \item \textit{Ethertap}
	  \item SLIP
	  \item \textit{Slirp}
	 \end{itemize}
	 \item Rede entre inst�ncias UML, utilizando um 
\textit{switch} virtual
	\end{itemize}

}

\frame{
	\frametitle{UML Switch}
	\begin{itemize}[<+-| alert@+>]
	 \item \textit{uml\_switch} programa integrante do pacote
UML permite interliga��o da m�quinas UML
	 \item Implementa um \textit{switch Ethernet} em
\textit{software}
	 \item As m�quinas se conectam ao switch e se comunicam por
meio de um arquivo UNIX soquetes no hospedeiro
	\end{itemize}

}

\frame{
	\frametitle{Redes virtuais UML}
	\pgfuseimage{vnuml}
}

\frame{
	\frametitle{Altera��es realizadas no UML Switch}
	Para permitir a simula��o de mobilidade nos cen�rios foram feitas as
seguintes altera��es no \textit{switch} virtual:
	\begin{itemize}[<+-| alert@+>]
	 \item Possibilidade de segmenta��o de redes (VLAN)
	 \item Mobilidade entre as VLANs
	 \item Servidor \textit{telnet} que tem a fun��o basica de
efetuar a mobilidade entre os n�s conectados ao switch. Seus comandos s�o:
	 \begin{itemize}
	  \item move $<$porta$>$ $<$VLAN$>$
	  \item list
	  \item quit
	 \end{itemize}
	\end{itemize}
}

\subsection{Interface Gr�fica de controle de terminais UML para o IP M�vel}
\frame{
	\frametitle{Uso amig�vel das m�quinas UML}
	\begin{columns}
    	 \column{11cm}
	\begin{itemize}[<+-| alert@+>]
	 \item Dificuldade na constru��o de cen�rios de rede
utilizando UML
	 \item Decidiu-se criar uma interface gr�fica amig�vel
	 \item O projeto GUML (Python) j� comtempla muitos dos 
recursos, por�m seu foco n�o � a constru��o de cen�rios de rede
	 \item Deriva��o do projeto e cria��o do GUML4MIP
	\end{itemize}
	 \column{1.5cm}
	\pgfuseimage{guml4mip_logo}
	\end{columns}
}

\frame{
	\frametitle{Recursos GUML4MIP}
	\begin{columns}
    	 \column{11cm}
	Principais recursos adicionados que melhoram o suporte a redes virtuais:
	\begin{itemize}[<+-| alert@+>]
	 \item Par�metros no arquivo de configura��o:
	 \begin{itemize}
	  \item fun��o da m�quina UML: roteador ou host
	  \item Endere�o da interfaces
	  \item VLAN da interface
	  \item Servi�os para iniciar no boot, por exemplo: mipl, radvd
	 \end{itemize}
	 \item Gera��o automatica de scripts de configura��o de
rede
	 \item Roteamento din�mico
	 \item Interface gr�fica para o \textit{uml\_switch}
	 \item Controle da mobilidade por meio da interface do
switch
	\item \textit{create\_fs} programa que constr�i
automaticamente o sistema de arquivos
	\end{itemize}
	 \column{1.5cm}
	\pgfuseimage{guml4mip_logo}
	\end{columns}
}

\frame{
	\frametitle{Interface GUML4MIP}
	\begin{columns}
    	 \column{11cm}
	\hspace{1cm}
	\pgfuseimage{guml4mip}
	 \column{1.5cm}
	\pgfuseimage{guml4mip_logo}
	\end{columns}

}

\frame{
	\frametitle{Interface gr�fica para o uml\_switch}
	\begin{columns}
    	 \column{11cm}
	Na interface gr�fica do \textit{uml\_switch} o usu�rio pode:
	\begin{itemize}[<+-| alert@+>]
	 \item Observar todas as m�quinas conectadas a ele
	 \item Qual VLAN cada interface esta associada
	 \item Efetuar a mobilidade clicando no n�mero da VLAN
	\end{itemize}
	\column{1.5cm}
	\pgfuseimage{guml4mip_logo}
	\end{columns}
	\vspace{0.5cm}
	\pgfuseimage{uml_switch}
}

\frame{
	\frametitle{Funcionamento do GUML4MIP}
	\begin{columns}
    	 \column{11cm}
	\hspace{1cm}
	\pgfuseimage{guml4mip_func}
	 \column{1.5cm}
	\pgfuseimage{guml4mip_logo}
	\end{columns}
}
\section{Implementa��o do HMIPv6 basendo-se no projeto MIPL}

\subsection{Mobile IP for Linux}

\subsection{Altera��es no MIPL}

% ----------------------------------------------------------------------- %
% Arquivo: cenarios.tex
% ----------------------------------------------------------------------- %

\chapter{Cen�rios de Testes}

\section{A Prepara��o dos Cen�rios}

\subsection{Cen�rios Estudados}
Ap�s os estudos bibliogr�ficos sobre o protocolo MIPv6 e a prepara��o da
Plataforma de Testes para Mobilidade, iniciou-se a etapa de defini��o e
implementa��o de cen�rios de testes.  Esta etapa teve por objetivo  analisar
algumas funcionalidades dos protocolos instalados, demonstrando
desta forma as potencialidades do ambiente desenvolvido.

Para analisar alguns mecanismos dos protocolos e testar os seus modos de
opera��o, dois cen�rios base foram definidos: cen�rio 1, com o MIPv6, e o
cen�rio 2, com o HMIPv6. Para o primeiro cen�rio, duas varia��es foram
realizadas. Na primeira, descrita na figura \ref{f_cenario1}, � testado o modo
de opera��o por tunelamento bi-direcional do MIPv6, sem a otimiza��o de rota. A
segunda varia��o testa o modo de opera��o por otimiza��o de roteamento. O
cen�rio 2, descrito na figura \ref{f_cenario2}, tem como objetivo verificar
minimamente o funcionamento do HMIPv6 em comunica��es onde os n�s
correspondentes n�o executam otimiza��o de roteamento.

Os arquivos para a configura��o dos cen�rios usados no GUML4MIP est�o
dispon�veis no diret�rio da documenta��o do projeto \cite{guml4mip}.

\subsection{Fundamentos para a An�lise dos Cen�rios}
Na an�lise dos cen�rios estudados foram focados os seguintes aspectos:
\begin{itemize}
  \item a sequ�ncia e a natureza das mensagens trocadas entre os v�rios n�s.
Neste ponto, foram utilizadas as sa�das geradas pelo programa \textit{tcpdump};
  \item as tabelas de roteamento e as regras de roteamento analisadas em
situa��es chave, por exemplo, antes e depois da mobilidade;
  \item os aspectos de desempenho e perdas de pacotes, tendo ci�ncia das
limita��es desta an�lise em um ambiente virtual. Os dados obtidos foram
tamb�m confrontados com os resultados esperados atrav�s de uma abordagem
anal�tica;
  \item a cria��o e a manipula��o dos t�neis entre o agente domiciliar e o n�
m�vel, antes e depois do movimento.
\end{itemize}

\subsection{Considera��es sobre o Tempo de lat�ncia do \textit{Handover}}
O processo de \textit{handover} acontece quando o n� m�vel muda seu ponto de
conex�o de uma sub-rede para outra. O tempo de lat�ncia envolvido neste processo
\cite{xavier} pode ser dividido em quatro fases:

\begin{enumerate}
\item \textbf{Detec��o de Movimento (\textit{TD})}: Em um cen�rio real, 
representa o tempo do \textit{handover} na camada de enlace at� o primeiro
RA. Na plataforma de testes constru�da n�o foi simulada a camada enlace e,
portanto, n�o se pode precisar com a exatid�o a lat�ncia envolvida no processo.
Por�m, para fins de estudo, considerar-se-� o tempo entre a entrega do �ltimo
pacote, de alguma conex�o, recebido na rede domiciliar ($t0$)e o primeiro RA na
rede visitada ($t1$).

\begin{equation}
 TD = t1 - t0
\label{eq.td}
\end{equation}

\item \textbf{Configura��o do CoA (\textit{TA})}: Tempo
entre o primeiro RA e o envio do BU ($t2$).

\begin{equation}
 TA = t2 -t1
 \label{eq.ta}
\end{equation}

\item \textbf{Registro com agente domiciliar (\textit{TR})}: Intervalo de tempo
entre o envio do BU ao agente domiciliar e o recebimento do BA($t3$).

\begin{equation}
 TR = t3 - t2
 \label{eq.tr}
\end{equation}

\item \textbf{Otimiza��o de Roteamento (\textit{TO})}: Intervalo de tempo entre
o envio das mensagens do RRP ($t4$) e o recebimento do BA do n� correspondente.

\begin{equation}
 TO = t4 - t3
 \label{eq.to}
\end{equation}

\end{enumerate}

Obviamente, pode-se estimar o tempo de lat�ncia do \textit{handover} com a
seguinte
f�rmula:

\begin{equation}
 TH = TD + TA + TR + TO
 \label{eq.th}
\end{equation}

\subsection{Metodologia para Realiza��o dos Cen�rios}
Nos cen�rios propostos, o n� m�vel inicialmente estar� em sua rede domiciliar e
depois ir� movimentar-se e visitar outras redes. Durante o procedimento de
movimento, testes ser�o realizados para a coleta de dados que ir�o ajudar na
an�lise do protocolo.

Em um teste, o n� correspondente come�a a enviar mensagens ICMPv6, por meio do
\textit{ping6}, para o n� m�vel:
\begin{verbatim}
cn# ping6 mn
\end{verbatim}

O n� m�vel executa o \textit{tcpdump} em modo \textit{verbose} (-vvv) para
realizar o monitoramento da troca de mensagens do funcionamento do MIPv6 durante
o processo de \textit{handover}:
\begin{verbatim}
mn# tcpdump -vvv
\end{verbatim}

Outro procedimento de teste, que � eventualmente realizado durante a execu��o
dos cen�ri-os, � uma conex�o utilizando o protocolo da camada transporte TCP.
Tal teste visa comprovar a transpar�ncia do processo de
mobilidade para as camadas superiores � de rede, al�m de permitir gerar um
fluxo de dados maior do que com as mensagens ICMPv6 e de possibilitar
uma melhor observa��o do impacto do protocolo MIPv6 sobre as conex�es.

Para realiza��o do teste � iniciado, no n� m�vel, um servidor TCP. Este 
servidor responde a requisi��o GET do protocolo \ac{HTTP}, al�m de ser capaz de
gerar
tr�fego. Para realizar esta tarrefa � utilizado a ferramenta \textit{netcat6},
da forma:
\begin{verbatim}
mn# echo "200 OK" > header; cat header /dev/zero | nc6 -l -p 80
\end{verbatim}

Esta linha significa:
\begin{description}
 \item[echo ``200 OK'' $>$ header:] cria um cabe�alho HTTP com a mensagem ``200 
OK'', para responder as requisi��es ao servidor;
 \item[nc6 -l -p 80:] inicia um soquete TCP que (-l) escuta a (-p 80) porta 80;
 \item[cat header /dev/zero:] concatena para a entrada padr�o do servidor o
arquivo contendo a mensagem HTTP e o dispositivo caracter /dev/zero que �
respons�vel por gerar o tr�fego.
 \end{description}

O n� correspondente ir� se conectar ao servidor utilizando da ferramenta 
\textit{wget}. Tamb�m no n� correpodente executamos a ferramenta 
\textit{speedometer} que � capaz de medir a taxa m�dia de dados de uma interface
de rede. A sa�da do \textit{speedometer} permite criar um gr�fico para
visualizar o efeito do MIPv6 em uma conex�o.

\begin{verbatim}
cn# wget -6 --limit-rate=1M http://mn -o /dev/null -O /dev/null &
cn# speedometer -p -i 0.1 -rx eth0
\end{verbatim}

Sobre os comandos executados no n� correspondente as seguintes considera��es:
\begin{itemize}
 \item O gerador produz uma taxa muito elevada de dados o
que ocasiona o travamento das m�quinas virtuais. Para resolver este problema a
taxa � limitada (--limit-rate=1M) para 1MBps;
 \item o \textit{speedtometer} vai monitorar a recep��o da interface
eth0 em intervalos de 0.1 segundos.
\end{itemize}

\subsection{Subs�dios para interpreta��o dos pacotes IPv6 no tcpdump}

Nas listagens do \textit{tcpdump}, utilizadas na se��es que se seguem, os
pacotes ser�o
mostrados da seguinte forma:

$<$estampa\_tempo$>$ IP6 ( .. next header $<$tipo\_next\_header$>$ ..)
$<$endere�o\_fonte$>$ \\ $<$endere�o\_destino$>$

No caso de t�neis, onde existe encapsulamento de IP sobre IP, o formato �
repetido e o campo \textit{next header} do primeiro n�vel indica que o pr�ximo
cabe�alho � um pacote IPv6. No segundo n�vel, o \textit{next header} � a carga
original transportada, por exemplo, um pacote ICMP6:

$<$estampa\_tempo$>$ IP6 ( .. next header IPv6 ..) 
        $<$endere�o\_fonte$>$ $<$endere�o\_destino$>$
        IP6 ( .. next header ICMP6 ..) 
        $<$endere�o\_fonte$>$ $<$endere�o\_destino$>$

� interessante lembrar que os RA s�o mensagens ICMP6, assim
como os \textit{echo-request} e \textit{echo-reply} produzidos pelo ping.

O \textit{tcmpdump} apresenta os cabe�alhos de mobilidade (BU e BA) com
\textit{next-header
unknown} no caso do BU e de \textit{source routing} no caso do BA. Entretanto, o
\textit{next-header} � detalhado na sequ�ncia mostrando o tipo e eventualmente o
DSTOPT (campo \textit{destination option}).


\section{Cen�rio 1 - Uso do Protocolo MIPv6}
\label{s_cenario1}

\subsection{Descri��o da Topologia}
O primeiro cen�rio de teste � formado por tr�s redes interligadas entre si e
cinco n�s. A sua topologia pode ser observada na figura \ref{f_cenario1}, onde o
numero nos bal�es cinza representam a VLAN no \textit{uml\_switch}.

\begin{figure}[!htpb]
	\centering
	\includegraphics[scale=.4]{figs/cenario1}
	\caption{Topologia do Cen�rio 1}
	\label{f_cenario1}
\end{figure}

Na rede com prefixo 2000:a::/64 est� o n� correspondente, com quem o n� m�vel
est� se comunicando. Na rede com o prefixo 2000:c::/64 est� presente um roteador
(HA) que oferece o servi�o de agente domiciliar ao n� m�vel. A rede com prefixo
2000:d::/64 � a rede que o n� m�vel deve visitar.

\subsection{Varia��o 1: com Tunelamento Bidirecional e Sem Otimiza��o de Rota}

\subsubsection{An�lise da Troca de Mesagens}
Os dados obtidos na sa�da do comando \textit{tcpdump} podem ser observados na
figura \ref{f_dump_1}. Analisando os dados recolhidos consegue-se observar:

\begin{figure}[!htpb]
	\centering
	\includegraphics[scale=.6]{figs/dump1}
	\caption{Cen�rio 1 : Captura de mensagens com Tunelamento Bidirecional}
 	\label{f_dump_1}
\end{figure}

\begin{enumerate}
 \item O n� m�vel recebe uma mensagem de RA do
roteador de acesso da rede visitada (src = fe80::fcfd:71ff:febf:e3e1), o que
indica que ocorreu mobilidade. Este movimento foi produzido pela comuta��o
da vlan 3 para a 4. Antes de realizar o movimento o n� m�vel recebia
\textit{echo request} do n� correspondente (src = 2001:a::fcfd:65ff:fe34:8a44) e
RA do Agente domiciliar (src = fe80::fcfd:f5ff:fe49:32bc);
 \item O n� m�vel realiza o teste de duplica��oo de endereco (DAD) do CoA gerado
com o prefixo da rede visitada (2001:d::fcfd:32ff:feff:42c9);
 \item O n� m�vel envia a mensagem de BU para o agente Domiciliar informando o
endere�o CoA e o seu endere�o domiciliar HoA. Note que o CoA � o endere�o fonte
do pacote e o HoA � informado na op��o DSTOPT;
 \item O agente domiciliar envia um BA, confirmando o registro;
 \item O n� m�vel volta a receber \textit{echo request} do n� correspondente,
s� que agora atrav�s do t�nel. Note os endere�os IP fonte e destinos deste
pacote. O primeiro n�vel IP do agente domiciliar(2001:c::1) para o CoA do n�
m�vel (2001:d::fcfd:32ff:feff:42c9). No segundo n�vel, do n� correspondente
(2001:a::fcfd:65ff:fe34:8a44) para o HoA (2001:c::2).

\end{enumerate}

\subsubsection{An�lise das Tabelas de Roteamento e de Regras de Roteamento}

Ap�s o n� m�vel deixar a rede domiciliar, podemos constatar mudan�as nas tabelas
de roteamento do n� m�vel e do agente domiciliar, que s�o mostradas nas 
respecitivas tabelas \ref{t_rot_mn1} e \ref{t_rot_ha1}.

\begin{table}[!htpb]
\centering
\begin{small}
  \setlength{\tabcolsep}{3pt}
\begin{tabular}{|c|c|c|c||}\hline \hline
\textbf{Destino} & \textbf{Via} & \textbf{Proximo
 salto} & \textbf{Considera��es} \\ \hline \hline
2001:a::/64 &	eth0	&	fe80::fcfd:dff:fe0c:a222 & \\
2001:b::/64 &   eth0	&	::  & \\
\textbf{2001:c::2} & \textbf{eth1} & \textbf{2001:c::2}  & Antes da mobilidade
do n� m�vel\\
2001:c::/64 &	eth1	&	::  & \\
2001:d::/64 &	eth0	&	fe80::fcfd:9fff:fe3d:cf8a  & \\ \hline
2001:a::/64 &	eth0	&	fe80::fcfd:dff:fe0c:a222  & \\
2001:b::/64 &	eth0	&	::  & \\
\textbf{2001:c::2} & \textbf{ip6tnl1} & \textbf{2001:c::2} & Ap�s a mobilidade
do n� m�vel \\
2001:c::/64 &	eth1	&	::  & \\
2001:d::/64 &	eth0	&	fe80::fcfd:9fff:fe3d:cf8a  & \\ \hline
\end{tabular}
\end{small}
\caption{Tabela principal de roteamento do agente domiciliar}
\label{t_rot_ha1}
\end{table}

\begin{table}[!htpb]
\centering
\begin{small}
  \setlength{\tabcolsep}{3pt}
\begin{tabular}{|c|c|c|c||}\hline \hline
\textbf{Destino} & \textbf{Via} & \textbf{Proximo
salto} & \textbf{Considera��es}\\ \hline \hline
2001:c::2	          & ip6tnl1		  & :: &  \\
2001:c::/64               & eth0		  & :: & Na rede domiciliar \\
default	                  & eth0		  & fe80::fcfd:f5ff:fe49:32bc &
\\ \hline
2001:c::2	          & ip6tnl1		  & :: &  \\
2001:d::/64               & eth0		  & :: & Na rede visitada \\
default		          & eth0	          & fe80::fcfd:71ff:febf:e3e1 &
\\ \hline
\end{tabular} 
\end{small}
\caption{Tabela principal de roteamento do n� m�vel}
\label{t_rot_mn1}
\end{table} 

Ap�s o agente domiciliar receber o \textit{Binding Update},
� adicionada a seguinte rota a sua tabela de roteamento, que todos os 
pacotes com destino ao endere�o domiciliar devem ser encaminhados para o t�nel.
Ap�s o n� m�vel receber o \textit{Router Advertisement}, � alterada a rota
padr�o, deletada a rota para rede 2001:c::/64 e adicionada uma rota para a rede 
2001:d::/64, onde a interface eth0 est� atrelada.

Para que o n� m�vel tenha os pacotes com destino ao n� correspondente e ao 
agente domiciliar encaminhados pelo t�nel, o que caracteriza o tunelamento
bidirecional, � adicionada regras na tabela 252, mostradas na tabela
\ref{t_route_252_mn}.

\begin{table}[!htpb]
\centering
\begin{small}
  \setlength{\tabcolsep}{3pt}
\begin{tabular}{|c|c|c|c||}\hline \hline
\textbf{Destino} & \textbf{Via} & \textbf{Proximo
salto} & \textbf{Considera��es}\\ \hline \hline
          *                 &   *     &     *     &  Na rede domiciliar\\ \hline
2001:a::fcfd:65ff:fe34:8a44 & ip6tnl1 & 2001:a::fcfd:65ff:fe34:8a44 & \\
2001:c::1                   & ip6tnl1 & 2001:c::1 &  Na rede visitada\\
default                     & ip6tnl1 & ::        & \\ \hline
\end{tabular} 
\end{small}
\caption{Tabela de Roteamento 252 do n� m�vel}
\label{t_route_252_mn}
\end{table}

No agente domiciliar � adicionada uma rota na tabela 252, para que os pacotes 
capturados com destino ao n� m�vel e origem no agente domiciliar.


\begin{table}[!htpb]
\centering
\begin{small}
  \setlength{\tabcolsep}{3pt}
\begin{tabular}{|c|c|c|c||}\hline \hline
\textbf{Destino} & \textbf{Origem} & \textbf{Via} & \textbf{Considera��es}\\
 \hline \hline
2001:c::2 & 2001:c::1 & eth1 & Ap�s a mobilidade do n� m�vel \\ \hline
\end{tabular}
\end{small}
\caption{Tabela de Roteamento 252 do agente domiciliar}
\label{t_route_252_ha}
\end{table}

Como existe mais de uma tabela de roteamento, para fazer a sele��o de qual 
tabela � mais adequada para encaminhar determinado pacote, � determinado
regras de roteamento. O MIPv6 adiciona um regra direcionando para tabela 252,
no n� m�vel e no agente domiciliar, mostradas nas tabelas \ref{t_rule_mn} e 
\ref{t_rule_mn}, respectivamente.

\begin{table}[!htpb]
\centering
\begin{small}
  \setlength{\tabcolsep}{3pt}
\begin{tabular}{|c|c|c||}\hline \hline
\textbf{Prioridade} & \textbf{Regra} & \textbf{A��o}
\\ \hline \hline
0:	& from all        &  lookup local \\
\textbf{1001}:	& \textbf{from 2001:c::2}  &  \textbf{lookup 252} \\
1002:	& from fe80::/64 &   lookup main \\
1002:	& from 2001:d::/64 & lookup main \\
1003:	& from 2001:c::2  &  blackhole \\
32766:	& from all        &  lookup main \\ \hline
\end{tabular} 
\end{small}
\caption{Regras de Roteamento do N� movel}
\label{t_rule_mn}
\end{table} 

\begin{table}[!htpb]
\centering
\begin{small}
  \setlength{\tabcolsep}{3pt}
\begin{tabular}{|c|c|c||}\hline \hline
\textbf{Prioridade} & \textbf{Regra} & \textbf{A��o}
\\ \hline \hline
0:	& from all        &  lookup local \\
\textbf{1005}:	& \textbf{from 2001:c::2 } &  \textbf{lookup 252} \\
32766:	& from all        &  lookup main \\ \hline
\end{tabular} 
\end{small}
\caption{Regras de Roteamento do Home Agente}
\label{t_rule_ha}
\end{table} 

\subsubsection{Detalhes da Forma��o de T�neis}
O t�nel no n� m�vel � criado antes de acontecer qualquer mobilidade. No agente
domiciliar o  t�nel com o n� m�vel � criado no recebimento do BU. As tabelas 
\ref{t_tun_ha1} e \ref{t_tun_mn1} apresentam o endere�amento do tunel, no HA e 
MN, antes e depois da mobilidade.

\begin{table}[!htpb]
\centering
\begin{small}
  \setlength{\tabcolsep}{3pt}
\begin{tabular}{|c|c|c|c|c||}\hline \hline
\textbf{Nome} & \textbf{Remoto} & \textbf{Local} & 
\textbf{Interface} & \textbf{Considera��es} \\ \hline \hline
ip6tnl1 & 2001:c::1 & 2001:c::2 & eth0 & Antes da mobilidade do n� m�vel \\
 \hline
ip6tnl1 & 2001:c::1 & 2001:d::fcfd:32ff:feff:42c9 & eth0 & Ap�s a mobilidade do 
n� m�vel \\ \hline
\end{tabular}
\end{small}
\caption{Descri��o dos \texttt{end-points} do tunel do n� m�vel}
\label{t_tun_mn1}
\end{table}

\begin{table}[!htpb]
\centering
\begin{small}
  \setlength{\tabcolsep}{3pt}
\begin{tabular}{|c|c|c|c|c||}\hline \hline
\textbf{Nome} & \textbf{Remoto} & \textbf{Local} & 
\textbf{Interface} & \textbf{Considera��es} \\ \hline \hline
*  & * & * & * & Antes da mobilidade do n� m�vel \\ \hline
ip6tnl1 & 2001:d::fcfd:32ff:feff:42c9 & 2001:c::1 & eth0 & Ap�s a mobilidade do 
n� m�vel \\ \hline
\end{tabular}
\end{small}
\caption{Descri��o dos \texttt{end-points} do tunel do agente domiciliar}
\label{t_tun_ha1}
\end{table}

%===========================================================================
\subsection{Varia��o 2: MIPv6 com Otimiza��o de Rota}

\subsubsection{An�lise da Troca de Mensagens}

Os dados obtidos na sa�da do comando \textit{tcpdump} podem ser observados na
figura \ref{f_dump_2}. Analisando os dados recolhidos consegue-se observar:

\begin{figure}[!htpb]
	\centering
	\includegraphics[scale=.6]{figs/dump2}
	\caption{Cen�rio 1 : Captura de mensagens com Otimiza��o de Rota}
 	\label{f_dump_2}
\end{figure}

\begin{enumerate}
%1
 \item O n� m�vel recebe uma mensagem de RA do 
roteador de acesso da rede visitada, o que indica que ocorreu mobilidade. Antes 
de realizar o movimento o n� m�vel respondia um \textit{echo request} do n� 
correspondente;
%2
 \item O n� m�vel realiza o teste de duplicacao de endereco (DAD) do CoA gerado
com o prefixo da rede visitada (2001:d::fcfd:32ff:feff:42c9);
%3
 \item O n� m�vel a mensagem de BU para o Agente domiciliar informando o atual 
endere�o. Note que o CoA � o endere�o fonte do pacote e o HoA � informado na 
op��o DSTOPT;
%4
 \item O agente domiciliar envia um BA, confirmando o registro;
%5
 \item O n� movel volta a receber \textit{echo request} do n� correspondente,
s� que ainda atrav�s do t�nel. A otimiza��o de rota s� ir� ocorrer ap�s o
processo de RRP. Note os endere�os IP fonte e destinos
deste pacote. O primeiro n�vel IP do agente domiciliar(2001:c::1) para n� m�vel
(2001:d::fcfd:32ff:feff:42c9). No segundo n�vel, do n� correspondente
(2001:a::fcfd:65ff:fe34:8a44).
%6
 \item O n� m�vel inicia o processo de \textit{ruturn routability} enviando
via agente domiciliar a mensagem de \textit{Home Test Init}(HoTi) ao n� 
correspondente, para adiquirir a \textit{home keygen token}.
%7
 \item O n� m�vel recebe a mensagem de \textit{Home Test}(HoT) do n�
correspondente, via agente domiciliar.
%8
 \item O n� m�vel da continua��o ao processo de \textit{ruturn routability}
enviando a mensagem de \textit{Care-of Test Init}(CoTi) ao n� correspondente,
para adiquirir a \textit{care-of keygen token}. Essa menssagem n�o passa pelo 
tunel, ela � endere�ada diretamente ao n� correspondente.
%9
 \item O n� m�vel recebe a mensagem de \textit{Care-of Test}(CoT) do n� 
correspondente, n�o passando pelo agente domiciliar.
%10
 \item O n� m�vel envia um \textit{Binding Update} ao n� correspondente com o
\textit{binding management key}(kbm), concluindo o processo de 
\textit{return routability}.
%11
 \item O n� m�vel recebe uma mensagem de \textit{echo request}, endere�ada ao 
CoA, n�o passando mais pelo agente domiciliar. Note que a op��o DSTOPT que tr�s
o endere�o domiciliar (2001:c::2) do n� m�vel.

\end{enumerate}

\subsubsection{Estado das Tabelas de Roteamento e de Regras}
Sobre as tabelas e as regras de roteamento, quando o n� m�vel e o n� 
correspondente est�o configurados para realizar a otimiza��o de rota, n�o existe
nenhuma altera��o em rela��o a varia��o 1.

\subsubsection{Detalhes da Forma��o de T�neis}
Sobre os tuneis, n�o h� nenhuma altera��o em relac�o ao cen�rio configurado com
tunelamento bidirecional sem otimiza��o de rota.

%=========================================================================

\section{Cen�rio 2 - Uso do Protocolo HMIPv6}
\label{s_cenario2}

\subsection{Descri��o da Topologia}
Com o fim de testar se a implementa��o experimental do HMIPv6 atende as
funcionalidades do protocolo, o cen�rio 2 foi montando com os seguintes 
componentes (ver figura \ref{f_cenario2}):

\begin{figure}[!htpb]
	\centering
	\includegraphics[scale=.3]{figs/cenario2}
	\caption{Topologia do Cen�rio 2}
 	\label{f_cenario2}
\end{figure}

\begin{itemize}
 \item Rede domiciliar com o prefixo 2001:d::/64 (VLAN 4), onde est�o o
agente domiciliar (HA) e o n� m�vel (MN), no in�cio do teste;
 \item Rede com o prefixo 2001:a::/64 (VLAN 1), onde est�o presentes um
roteador de acesso (AR1) e o n� correspondente (CN). Este �ltimo ir� se
comunicar com o n� m�vel durante o teste;
 \item Um dom�nio formado pelo roteador MAP, que executa o \textit{daemon} MIPL
na
fun��o de agente domiciliar, e dois roteadores de acesso (AR2 e AR3), que
possibilitam o teste de mobilidade em um mesmo dom�nio. Fazem parte
deste dom�nio as sub-redes com os prefixos: 2001:c::/64 (VLAN 3),
2001:e::/64 (VLAN 5) e 2001:f::/64 (VLAN 6).
\end{itemize}

Durante a execu��o do cen�rio:
\begin{enumerate}
 \item O n� m�vel iniciar� na rede domiciliar se comunicando via \textit{ping6}
com o n� correspondente.
 \item Uma situa��o de mobilidade ser� ocasionada e o n� ir� migrar para a
VLAN 5, dom�nio do MAP, caracterizando uma mobilidade global.
 \item Na segunda situa��o de mobilidade o n� ir� migrar para a VLAN 6
caracterizando uma mobilidade local.
\end{enumerate}

\subsection{An�lise da Troca de Mensagens}

\subsubsection{Mobilidade Global}
Durante a primeira situa��o de mobilidade do cen�rio 2, uma mobilidade global,
os dados coletados com a ferramenta \textit{tcpdump} est�o dispon�veis na figura
\ref{f_dump_inter}. Por meio deles pode-se observar que:

\begin{figure}[!htpb]
	\centering
	\includegraphics[scale=.55]{figs/dump_inter}
	\caption{Cen�rio 2: \textit{Logs} Mobilidade Global}
 	\label{f_dump_inter}
\end{figure}

\begin{enumerate}
 \item Inicialmente o n� m�vel est� na sua rede domiciliar recebendo 
\textit{echo request}s do n� correspondente e mensagens RA do
agente domiciliar. Pode-se evidenciar este fato pelo campo
src=fe80::fcfd:f5ff:fe49:32bc do RA que � o endere�o link local do agente
domiciliar. A mensagem 1 evidencia a mobilidade do n� m�vel pois ele passa a
receber mensagens RA do roteador de acesso AR2 (src=fe80::fcfd:71ff:febf:e3e1,
endere�o link local de AR2) com a op��o de MAP propagada pelo roteador MAP.
Esta op��o n�o � mostrada pelo tcpdump. Com este RA o n� m�vel forma os
endere�os LCoA(2001:e::fcfd:32ff:feff:42c9) e RCoA(2001:e::fcfd:32ff:feff:42c9);
 \item O n� m�vel realiza o DAD do LCoA
gerado com o prefixo da rede visitada (endere�o fe80::fcfd:32ff:feff:42c9); 
 \item O n� m�vel envia a mensagem de BU para o MAP (dst=2001:c::1) informando a
dupla
(LCoA,RCoA). Note que LCoA � o  endere�o fonte do pacote e o RCoA � informado
na op��o DSTOPT;
\item O n� m�vel envia a mensagem de BU para o Agente Domiciliar (dst=2001:d::1)
informando a dupla (RCoA,HoA). Note que RCoA � o endere�o fonte do pacote e o
HoA � informado na op��o DSTOPT;
 \item O n� m�vel recebe a mensagem de BA do MAP como confirma��o de registro
bem sucedido;
 \item O n� m�vel recebe a mensagem de BA do agente domiciliar, como confirma��o
de registro bem sucedido, via MAP, de forma tunelada. Note no primeiro n�vel os
endere�os IP do MAP (2001:d::1) e IP LCoA (2001:e::fcfd:32ff:feff:42c9). Na
sequ�ncia tamb�m chega um pacote provindo do n� correspondente que, embora
n�o expl�cito pelo \textit{tcpdump}, provavelmente pertence a um \textit{echo
request}
duplamente tunelado. Note os endere�os IPs fonte e destino destes pacotes: no
primeiro n�vel IP do MAP para IP LCoA e, no segundo n�vel, IP do
Agente domiciliar para IP RCoA;
 \item Envia um \textit{echo reply} ao n� correspondente, com destino ao agente
domiciliar, com a mensagem tunelada ao n� correspondente, via MAP.
\end{enumerate}

\subsubsection{Mobilidade Local}
Na segunda situa��o de mobilidade do cen�rio 2, uma mobilidade local, as
mensagens trocadas pelo n� m�vel est�o dispon�veis na figura \ref{f_dump_intra}.
Por meio delas pode-se observar que:

\begin{enumerate}
 \item � realizada a mobilidade e passa receber mensagens RA do roteador de
acesso AR3 ainda sobre cobertura do mesmo MAP.
 \item Realiza o DAD do novo LCoA gerado.
 \item Envia a mensagem de BU para o MAP com os seguintes campos:
\begin{itemize}
 \item Care-of-Address = LCoA;
 \item Endere�o domiciliar = RCoA, gerado a partir da op��o de MAP da mensagem
RA.
\end{itemize}
 Como � uma mobilidade local n�o � necess�rio enviar mensagem ao agente 
domiciliar.
 \item Recebe a mensagem de BA do MAP como confirma��o de registro bem
sucedido.
 \item Recebe um \textit{echo reply} do n� correspondente, via MAP com a 
mensagem tunelada do agente domiciliar.
\end{enumerate}
\begin{figure}[!htpb]
	\centering
	\includegraphics[scale=.55]{figs/dump_intra}
	\caption{Cen�rio 2: \textit{Logs} Mobilidade Local}
 	\label{f_dump_intra}
\end{figure}

\subsection{Estado das Tabelas de Roteamento e de Regras}
Sobre as tabelas e as regras de roteamento, a implementa��o do HMIpv6 n�o faz
nenhuma altera��o. As tabelas de roteamento do cen�rio 2 se diferem muito pouco
das do cen�rio 1, tanto na mobilidade global como na local. As tabelas 252
(\ref{t_c2_route_252}) e a principal (\ref{t_c2_route_main}), ap�s a primeira
situa��o de mobilidade est�o dispon�veis logo abaixo. E seguem algumas
considera��es sobre as regras e tabelas de roteamento do cen�rio 2:

\begin{table}[!htpb]
\centering
\begin{small}
  \setlength{\tabcolsep}{3pt}
\begin{tabular}{|c|c|c|c||}\hline \hline
\textbf{Destino} & \textbf{Via} & \textbf{Proximo
salto}\\ \hline \hline
2001:a::fcfd:65ff:fe34:8a44 & ip6tnl1 & 2001:a::fcfd:65ff:fe34:8a44 \\
2001:d::1                   & ip6tnl1 & 2001:d::1\\
default                     & ip6tnl1 & :: \\ \hline
\end{tabular} 
\end{small}
\caption{Cen�rio 2: Tabela de Roteamento 252}
\label{t_c2_route_252}
\end{table}


\begin{table}[!htpb]
\centering
\begin{small}
  \setlength{\tabcolsep}{3pt}
\begin{tabular}{|c|c|c|c||}\hline \hline
\textbf{Destino} & \textbf{Via} & \textbf{Proximo
salto} \\ \hline \hline
2001:d::fcfd:32ff:feff:42c9 & ip6tnl2		  & :: \\
2001:e::2	          & ip6tnl1		  & :: \\
2001:10::/64              & eth0		  & :: \\
default		          & eth0	          & fe80::fcfd:53ff:fe16:31aa \\
\hline
\end{tabular} 
\end{small}
\caption{Cen�rio 2: Tabela principal de roteamento}
\label{t_c2_route_main}
\end{table} 

\begin{itemize}
 \item antes da primeira situa��o de mobilidade, as tabelas e regras de
roteamento do cen�rio 2 s�o iguais as cen�rio 1 antes da mobilidade;
 \item ap�s a mobilidade:
\begin{itemize}
 \item as regras de roteamento s�o as mesmas que as do cen�rio com MIPv6;
 \item a tabela 252 se mant�m a mesma que a do cen�rio com MIPv6.
 \item a rota padr�o � alterada para o roteador de acesso AR3 e depois na
segunda situa��o de mobilidade para o roteador AR4;
 \item adicionada a rota que pacotes com destino ao RCoA s�o via ip6tnl2;
 \item as outras rotas se mant�m as mesmas que as de um cen�rio com MIPv6.
\end{itemize}
\end{itemize}

\subsection{Detalhes da Forma��o de T�neis}
Os detalhes da forma��o dos t�neis do cen�rio 2 diferem um pouco do cen�rio 1 a
implementa��o do HMIPv6 � a respons�vel. Nas tabelas \ref{t_tun_inter} e
\ref{t_tun_intra} � poss�vel ver os t�neis presentes no n� m�vel ap�s as
mobilidades global e local. As principais diferen�as do cen�rio 1 s�o:

\begin{itemize}
 \item a cria��o do t�nel (ip6tnl2) entre o LCoA do n� m�vel 
(2001:e::fcfd:32ff:feff:42c9) e o MAP (2001:c::1). Este t�nel � necess�rio, pois
os pacotes oriundos do Agente domiciliar ser�o endere�ados agora ao
RCoA(2001:c::fcfd:32ff:feff:42c9). Como h� no MAP um registro do LCoA-RCOA, este
t�nel passa a encaminhar os pacotes do HA ao n� m�vel;
 \item para que os pacotes vindos via MAP destinados ao endere�o domiciliar 
(2001:d::2) sejam entregues ao n� m�vel � necess�rio a altera��o dos pontos
finais do t�nel entre o n� m�vel e o Agente domiciliar, para o RCoA e o endere�o
do agente domciliar, e a interface que este t�nel est� ligado agora deve ser o
outro t�nel criado entre n� m�vel e o MAP.
\end{itemize}

Os t�neis dos MAP e agente domiciliar se mant�m iguais aos de um cen�rio
rodando o MIPv6.

\begin{table}[!htpb]
\centering
\begin{small}
  \setlength{\tabcolsep}{3pt}
\begin{tabular}{|c|c|c|c|c||}\hline \hline
\textbf{Nome} & \textbf{Remoto} & \textbf{Local} & 
\textbf{Interface} \\ \hline \hline
ip6tnl1 & 2001:d::1 & 2001:c::fcfd:32ff:feff:42c9 & ip6tnl2 \\
 \hline
ip6tnl2 & 2001:c::1 & 2001:e::fcfd:32ff:feff:42c9 & eth0  \\ \hline
\end{tabular}
\end{small}
\caption{Cen�rio 2: T�neis Mobilidade Global}
\label{t_tun_inter}
\end{table}

Ap�s a mobilidade local verificamos que os dois t�neis (ver \ref{t_tun_intra})
continuam presentes no n� m�vel. Por�m, o ponto local do t�nel \textit{ip6tnl2}
foi alterado para o novo LCoA (2001:f::fcfd:32ff:feff:42c9) gerado na sub-rede
do AR3.

\begin{table}[!htpb]
\centering
\begin{small}
  \setlength{\tabcolsep}{3pt}
\begin{tabular}{|c|c|c|c|c||}\hline \hline
\textbf{Nome} & \textbf{Remoto} & \textbf{Local} & 
\textbf{Interface} \\ \hline \hline
ip6tnl1 & 2001:d::1 & 2001:c::fcfd:32ff:feff:42c9 & ip6tnl2 \\
 \hline
ip6tnl2 & 2001:c::1 & 2001:f::fcfd:32ff:feff:42c9 & eth0  \\ \hline
\end{tabular}
\end{small}
\caption{Cen�rio 2: T�neis Mobilidade Local}
\label{t_tun_intra}
\end{table}

\section{An�lise dos Tempos de \textit{Handover}}
O tempo do handover � um dos maiores problemas do MIPv6, esta se��o do trabalho
pretende fazer uma an�lise deste processo apontar as causas e poss�veis
solu��es. Como base para estudo dos tempos de \textit{handover} o cen�rio 1 com
tunelamento bidirecional foi utilizado.

Para verificar a perda na taxa de transmiss�o em um processo de
\textit{handover}, foi realizado o segundo teste especificado na se��o sobre a
metodologia na realiza��o dos cen�rios. Utilizando a ferramenta \textit{GNUPLOT}
com a sa�da do teste foi gerado o gr�fico que pode ser observado na figura
\ref{f_banda}.

\begin{figure}[!htpb]
	\centering
	\includegraphics[scale=.6]{figs/banda}
	\caption{Taxa de transmiss�o em \textit{Handover}}
	\label{f_banda}
\end{figure}

A partir  da figura � poss�vel concluir que cada \textit{handover} � de
aproximadamente 3 segundos, h� perdas consider�veis de pacotes que prejudicariam
muito comunica��es interativas como videoconfer�ncia e \textit{VoIP}, por�m,
navega��o em p�ginas da \textit{Internet} e visualiza��o de \textit{e-mail} a
instabilidade � toler�vel.

Como cen�rio estudo foi simulado todos os tempos medidos s�o estimados. Por�m,
segundo algumas bibliografias estudadas que realizaram experimentos fisicamente
os tempos medidos na simula��o est�o dentro do esperado.

Na tabela \ref{t_handover}, encontram-se os tempos das fases do processo de
\textit{handover} referentes ao cen�rio estudado.

\begin{table}[!htpb]
\centering
\begin{small}
  \setlength{\tabcolsep}{3pt}
\begin{tabular}{|c|c|c|}\hline
\textbf{Fase} & \textbf{Tempo (ms)} & \textbf{Media
\%} \\ \hline
\textit{TD} & 721 & 23,11 \\ \hline
\textit{TA} & 1371 & 43.92\\ \hline
\textit{TR} & 1029 & 32.97\\ \hline
\textit{TH} & 3121 & 100 \\ \hline
\end{tabular} 
\end{small}
\caption{Lat�ncia no \textit{Handover} do cen�rio estudado}
\label{t_handover}
\end{table} 

Algumas tentativas podem ser feitas para tentar diminuir a lat�ncia do
\textit{handover} nas fases de detec��o de movimento e de configura��o do
CoA.

\begin{figure}[!htpb]
	\centering
	\includegraphics[scale=.6]{figs/radvd}
	\caption{Diferentes intervalos entre as mensagens \textit{Router
Advertisement}}
	\label{f_radvd}
\end{figure}

Na tentativa de diminuir o tempo da fase de detec��o de movimento, podemos
diminuir o intervalo entre as mensagens de RA do
roteador presente na rede que ir� ser visitada pelo n� m�vel. Pois, � por meio
desta mensagem que o n� m�vel detecta o movimento e desencadeia o processo de
registro.

Para verificar se h� uma melhora no processo de handover. O cen�rio
estudado foi repetido variando o intervalo entre as mensagens de
\textit{Router Advertisement}. O resultado deste teste esta dispon�vel na figura
\ref{f_radvd}.

Com os dados levantados, � poss�vel perceber que o intervalo entre as
mensagens de RA esta diretamente ligado com a perda
de pacotes durante o \textit{handover}. Por�m, esta n�o � uma boa pr�tica para
tentar amenizar a lat�ncia do handover, pois um numero muito grande de mensagen
\textit{multicasting} inundaria a rede prejudicando a perfomance principalmente
em redes sem fio.

No processo de forma��o de um novo CoA o DAD � necess�rio
para se certificar que o endere�o formado � exclusivo. Para o teste ser bem
sucedido nenhum n� vizinho deve enviar um \ac{NA}, em
resposta ao teste, ou NS, como o mesmo endere�o em
quest�o, em um per�odo de \textit{1000ms} segundo a RFC 2462 \cite{rfc2462}.
Este tempo de espera compromete a fase de configura��o do CoA.

\begin{figure}[!htpb]
	\centering
	\includegraphics[scale=.6]{figs/banda2}
	\caption{Taxa de transmiss�o em \textit{Handover} sem DAD e baixo
intervalo de RA}
	\label{f_banda2}
\end{figure}

Existe uma vari�vel chamada \textit{dad\_transmits} no \textit{kernel} do Linux
que � usada para configurar o DAD em um n�. Com a inten��o de diminuir a
lat�ncia na fase de configura��o do CoA, foi configurado a
vari�vel com o valor 0, isso significa que o procedimento de DAD � cancelado, e
foi alterado o valor do intervalo das mensagens de RA
no roteador de acesso da rede visitada para 0.03 segundos. Ent�o foi repetido o
teste. O resultado pode ser observado na figura \ref{f_banda2}.

Apartir, deste teste � poss�vel perceber uma queda de 3 segundos para
aproximadamente 1 segundo no tempo do \textit{handover} com as mudan�as
realizadas. Entretanto h� um protocolo dedicado a acelerar o processo de
handover o \ac{FMIPv6}. A id�ia do FMIPv6 � providenciar
informa��o relativa � camada de rede, com o objetivo de prever ou responder
prontamente a um evento de handover \cite{rfc4068}.

\section{Conclus�es sobre a An�lise dos Cen�rios}
Com a realiza��o dos experimentos podemos constatar o funcionamento do protocolo
e perceber continua��o da comunica��o entre os pontos comunicantes ap�s a
mobilidade de forma transparente as camadas superiores a de rede, observar todas
as mensagens trocadas no processo e estimar o tempo de lat�ncia com a
mobilidade.

O primeiro resultado que se pode obter com a realiza��o do Cen�rio 1, foi o
perfeito funcionamento do MIPv6, ocorreram algumas perdas de pacotes, mas o n�
m�vel conseguiu continuar sendo alcan�ado pelo seu endere�o domiciliar mesmo n�o
estando em sua rede domiciliar.

Na realiza��o do segundo cen�rio, foi poss�vel verificar minimamente o
funcionamento
do HMIPv6, claro que os problemas ainda existentes na implementa��o n�o
permitiram comprovar uma melhora no \textit{handover}. Por�m, pode-se ver um
avan�o j� que n�o tinhamos nenhuma vers�o recente deste protocolo.

Com a realiza��o dos cen�rios � poss�vel verificar que o \textit{handover} � o
� o grande problema MIPv6, por isso, a necessidade de outras extens�es para
trabalharem juntos com o MIPv6 como: o HMIPv6 e FMIPv6.
\section{Conclus�o}

\frame{
\frametitle{Trabalhos Futuros}

\begin{itemize}
\item Concep��o de um simulador da rede sem fio nas m�quinas UML, de forma
similar ao MobUML, possibilitando a an�lise dos problemas associados ao enlace e
seus impactos na camada de rede;
\item Implementa��o de modelos de mobilidade
cl�ssicos, de forma a produzir
movimentos circulares, retil�neos, randomizados, ou mesmo, a acoplagem a um
simulador de movimentos urbanos, tal como o \cite{krajzewicz};
\item Incrementos na linguagem de descri��o da UML, de forma a incorporar
aspectos de configura��o do IPv4 e facilitar a incorpora��o de novos protocolos
e \textit{deamons};
\end{itemize}

}

\frame{
\frametitle{Trabalhos Futuros}
\begin{itemize}
\item Melhorias na configura��o dos terminais virtuais e consoles, de forma a
obter Uma melhor visualiza��o do sistema por parte do usu�rio.
\item Desenvolvimento de uma interface gr�fica para a configura��o das
m�quinas
UML;
\item Cria��o autom�tica de uma conex�o virtual entre a m�quina hospedeira e
as
m�quinas virtuais; 
\item Melhorias no instalador do sistema.
\end{itemize}
}

\section*{Bibliografia}

\frame{
	% estilo para aparecer um icone de um livro
	%\beamertemplatebookbibitems
	% estilo para aparecer um icone de um paper =)
	\beamertemplatearticlebibitems
	\begin{thebibliography}{1}

	\bibitem{rfc3775}
	JOHNSON, D. ; PERKINS, C.; ARKKO, J.
	\newblock{\em Mobility Support in IPv6. RFC 3775}
	\newblock{\url http://www.ietf.org/rfc/rfc3775.txt}

	\bibitem{rfc4140}
	SOLIMAN, H.; CASTELLUCCIA, C.; EL MALKI, K.; BELLIER, L.
	\newblock {\em Hierarchical Mobile IPv6 Mobility Management, RFC 4140}
	\newblock{\url http://www.ietf.org/rfc/rfc4140.txt}

	\bibitem{rfc4068}
	KOODLI, R. 
	\newblock {\em Fast Handovers for Mobile IPv6, RFC 4068}
	\newblock{\url http://www.ietf.org/rfc/rfc4068.txt}
	
	\bibitem{mipl}
	NUORVALA, V.; PETANDER, H.; TUOMINEN,
	\newblock MIPL PROJECT. Mobile IPv6 for Linux, Version 2.0 RC1. 2004
	\end{thebibliography}
}

\end{document}
